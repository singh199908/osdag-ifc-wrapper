\documentclass[11pt, letterpaper]{article}

% Packages
\usepackage[utf8]{inputenc}
\usepackage[T1]{fontenc}
\usepackage[margin=1in]{geometry}
\usepackage{titlesec}
\usepackage{hyperref}
\usepackage{xcolor}
\usepackage{listings}

% Custom Colors
\definecolor{primary}{RGB}{0, 51, 102}
\definecolor{codegray}{rgb}{0.5,0.5,0.5}

% Formatting
\titleformat{\section}{\large\bfseries\color{primary}}{}{0em}{}[\titlerule]
\titlespacing*{\section}{0pt}{12pt}{6pt}

\lstset{
    basicstyle=\ttfamily\small,
    breaklines=true,
    postbreak=\mbox{\textcolor{red}{$\hookrightarrow$}\space},
}

\begin{document}

% Title
\begin{center}
    {\huge \textbf{Osdag IFC Wrapper - Methodology Report}} \\
    \vspace{5pt}
    \textbf{Internship Screening Task Submission} \\
    \vspace{5pt}
    Akshat Grover $|$ VIT Bhopal \\
    \href{https://github.com/singh199908/osdag-ifc-wrapper}{github.com/singh199908/osdag-ifc-wrapper}
\end{center}

\vspace{10pt}

\section{Introduction}
This report details the development of an IFC (Industry Foundation Classes) wrapper for Osdag, an open-source steel design software. The utility provides a bridge between PythonOCC topological shapes and the BIM ecosystem, enabling seamless export of 3D structural models.

\section{Methodology}

\subsection{Architectural Approach}
The core objective was to create a non-intrusive hook into Osdag's existing CAD generation logic. The wrapper accepts a dictionary of component names and their corresponding \texttt{TopoDS\_Shape} objects. This allows any Osdag module (e.g., Beam-to-Column End Plate, Tension Member) to use the wrapper with minimal changes to the source code.

\subsection{Geometric Extraction Strategy}
A significant challenge was the lack of reliable high-level IFC serialization in standard PythonOCC environments. To solve this, I developed a \textbf{Manual Triangulation Engine}:
\begin{enumerate}
    \item \textbf{Meshing:} Using \texttt{BRepMesh\_IncrementalMesh}, the topological shapes are converted into a discrete triangular mesh.
    \item \textbf{Data Recovery:} The engine iterates through the faces of the shape, extracting nodes and triangles while respecting local coordinate transformations.
    \item \textbf{BIM Reconstruction:} The extracted data is then mapped to \texttt{IfcFacetedBrep} entities within an IFC4 schema using \texttt{IfcOpenShell}.
\end{enumerate}

\subsection{Semantic Mapping}
To ensure the exported models are functional in BIM software (like BIMvision or Revit), a semantic inference engine classifies Osdag components into standard IFC classes:
\begin{itemize}
    \item Beams $\rightarrow$ \texttt{IfcBeam}
    \item Columns $\rightarrow$ \texttt{IfcColumn}
    \item Plates/Gussets $\rightarrow$ \texttt{IfcPlate}
    \item Bolts $\rightarrow$ \texttt{IfcMechanicalFastener}
\end{itemize}

\section{Challenges Faced}

\subsection{Library Version Discrepancies}
Standard builds of \texttt{IfcOpenShell-Python} often lack the \texttt{geom} serialize/deserialize hooks.
\textbf{Solution:} Developing a low-level triangulation bridge bypassed these dependencies, ensuring the wrapper works with a basic \texttt{pip install ifcopenshell}.

\subsection{BIM Viewer Visibility}
Initial exports were empty due to missing spatial context.
\textbf{Solution:} Explicitly defining an \texttt{IfcGeometricRepresentationContext} and linking it to the \texttt{IfcProject} corrected the camera extents and visibility in third-party viewers.

\section{References}
\begin{enumerate}
    \item Osdag Developer Documentation: \href{https://osdag.fossee.in/}{https://osdag.fossee.in/}
    \item IfcOpenShell-Python Official API: \href{https://ifcopenshell.github.io/}{https://ifcopenshell.github.io/}
    \item pythonOCC Data Exchange Demos: \href{https://github.com/tpaviot/pythonocc-demos}{https://github.com/tpaviot/pythonocc-demos}
    \item buildingSMART IFC4 Schema Specifications.
\end{enumerate}

\end{document}
